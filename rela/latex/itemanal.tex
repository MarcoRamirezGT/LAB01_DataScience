\HeaderA{itemanal}{Item analysis function}{itemanal}
\keyword{manip}{itemanal}
\keyword{misc}{itemanal}
\begin{Description}\relax
This function enables the user to evaluate the functioning of two or more items as a 
coherent scale. Among the traditionally used Cronbach's Alpha the function also produces standardized estimates 
as well as Duhachek and Iacobucci's (2004) proposed standrad errors and respective confidence intervals for the
reliability coefficients. Further, the function provides a bootstrap estimate of the convidance interval of both 
the regular and standardized alpha values. The function's output can be used to produce three plots: 
A density plot of the alpha and standardized alpha bootstrap simulations, a line plot of the "if item deleted" 
alpha values by each item and a star plot for all of the items for their respective "if item deleted" scale values. 
These plots can be called by submitting the output object of the function to the plot() statement. Similarly, the
output can be viewed in an abbreviated form by submitting the output object to the summary() function.
\end{Description}
\begin{Usage}
\begin{verbatim}itemanal(object, SE.par = 1.96, boots = 1000)\end{verbatim}
\end{Usage}
\begin{Arguments}
\begin{ldescription}
\item[\code{object}] Numeric dataset (usually a coerced matrix from a prior data frame) containing all items of the scale. The dataset is arranged observations (rows) by measure items (columns). 
\item[\code{SE.par}] Confidence interval corresponding Z-score. By default set to the 95 \% confidence interval Z-score of 1.96. 
\item[\code{boots}] Number of boot strap samples computed. By default 1,000 simulations are estimated. 
\end{ldescription}
\end{Arguments}
\begin{Details}\relax
The function is sensitive to the how the dataset was compiled. Using the cbind function will often return a matirx that appears numeric but in reality functions as a numeric compiled list. If system error messages occur try transforming the dataset using as.matrix(data.frame(your.dataset)).
\end{Details}
\begin{Value}
Output consists of a list with the following values:\\

\begin{ldescription}
\item[\code{Variables}] General information about the entered items such as item type, number of cases used in the analysis, minimum, maximum values and item sum. 
\item[\code{Tendency}] Contains the measures of central tendancy: The respective item mean, median, standard deviation (SD), standard error of the mean (SE.mean) , lower and upper values of a 95 \% confidence interval of the mean and item variance. 
\item[\code{Skewness}] Skewness, standard error of the skew, lower and upper values of the skew. 
\item[\code{Kurtosis}] Kurtosis, standard error of kurtosis as well as its respective 95 \% confidence interval values. 
\item[\code{Covariance}] The covariance matrix of all items in the dataset. 
\item[\code{Correlation}] The correlation matrix of all submitted items. 
\item[\code{Alpha}] The number of items in the scale as well as the covariance based Cronbach's alpha estimate. 
\item[\code{Conf.Alpha}] Standard error of Cronbach's alpha with the associated lower and upper bound confidence interval values. 
\item[\code{Bootstrap.Simmulations}] The regular (covariance based) alpha bootstrap simulated estimates. 
\item[\code{Alpha.Bootstrap}] Bootstrap mean, standard error and confidence interval lower and upper limits. 
\item[\code{Std.Alpha}] The number of items in the scale as well as the correlation based Cronbach's alpha estimate. 
\item[\code{Conf.Std.Alpha}] Standard error of the standardized Cronbach's alpha with the associated lower and upper bound confidence interval values. 
\item[\code{Bootstrap.Std.Simmulations}] The standardized (correlation based) alpha bootstrap simulated estimates. 
\item[\code{Alpha.Std.Bootstrap}] Standardized bootstrap mean, standard error and confidence interval lower and upper limits. 
\item[\code{Scale.Stats}] Changes in scale statistics upon deletion of any one item in the scale. Contains scale mean and variance. 
\item[\code{Alpha.Stats}] Changes in the scale's reliability estimate alpha upon deletion of any one item. Contains, corrected total item correlation, squared multiple correlation, adjusted alpha statistic without given item. 
\item[\code{call}] Submitted function call. 
\end{ldescription}
\end{Value}
\begin{Note}\relax
Under the current version of this function/package missing data is deleted listwise. Consequently only full cases are used in determining scale reliability. Also, note that the legends for the plots are placed interactively. 
Furthermore, the default plot function for the itemanal() object uses a windows() statement to produce several 
plots. This prevents prior plots from being replaced at the same time allowing for numerous plots to be produced.
However, this option may not be fully functional in MAC and Linux environemnts since the function calls on a
windows metafile.
\end{Note}
\begin{Author}\relax
Michael Chajewski ( \url{http://www.chajewski.com} )
\end{Author}
\begin{References}\relax
Cronbach, L. J. (1951). Coefficient alpha and the internal structure of tests. \emph{Psychometrika, 16(3)}, 297-334. \\
\\
Duhachek, A. \& Iacobucci, D. (2004). Alpha's standard error (ASE): An accurate and precise confidence interval estimate. \emph{Journal of Applied Psychology, 89(5)}, 792-808. \\
\\
Kim, J., \& Mueller, C. W. (1978). \emph{Introduction to factor analysis: What it is and how to do it}. SAGE Publications: Newbury Park, CA. \\
\\
Nunnally, J. C. \& Bernstein, I. H. (1994). \emph{Psychometric theory} (3 ed.). McGraw-Hill: New York, NY. \\
\\
Kaiser, H. F. \& Cerny, B. A. (1979). Factor analysis of the image correlation matrix. \emph{Educational and Psychological Measurement}, 39, 711-714.\\
\\
Pett, M. A., Lackey, N. R., \& Sullivan, J. J. (2003). \emph{Making sense of factor analysis: The use of factor analysis for instrument development in health care research}. SAGE Publications: Thousand Oaks, CA.
\end{References}
\begin{Examples}
\begin{ExampleCode}

library(rela)

Belts <- Seatbelts[,1:7]
Belts.item <- itemanal(Belts)
Belts.item
summary(Belts.item)
plot(Belts.item)

Belts2 <- Belts[,-5]
Belts2 <- Belts2[,-5] 
Belts.item2 <- itemanal(Belts2)
Belts.item2
summary(Belts.item2)
plot(Belts.item2)
\end{ExampleCode}
\end{Examples}

